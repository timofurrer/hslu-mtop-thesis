\chapter{Einleitung}
\label{cha:Einleitung}

\section{Aufgabenstellung}

Im Rahmen der \textit{Mainframe Topics} Blockwoche der Hochschule Luzern muss eine Abschlussarbeit,
ueber eines von vielen vorgegebenen Themen, geschrieben werden.

Das ausgewählte Thema dieser Arbeit ist \textit{Linux on z; Was unterscheidet Linux on z zu einem Linux auf x86}.
In den nachfolgenden Kapitel dieser Arbeit gehts es darum wichtige Unterschiede zwischen einem Linux, welches auf Standard x86 Rechnern und einem Linux welches auf einem IBM Mainframe laeuft aufzuzeigen.

Die erste Frage, die es zu klaeren gilt, ist ob es ueberhaupt noch Unterschiede gibt. Ist der Linux Kernel derselbe auf einem x86 und einem s/390x Mainframe? Oder entwickelt und liefert IBM einen \textit{geforkten}\footnote{Als Fork bezeichnet man eine abgeleitete Software, die eigenstaendig weiterentwickelt wird} Linux Kernel fuer ihre Mainframes?

Weitere Fragen die in dieser Arbeit behandelt werden:
\begin{description}
    \item[Workload]{Wie gut ist die Workload Verteilung von Linux auf einem x86 Rechner zu einem Mainframe?}
    \item[Konsolidierung]{Wie viele x86 Rechner mit Linux braucht es im Gegensatz zu einem Mainframe mit Linux?}
\end{description}

Nebst diesen Fragen sollen auch zusaetzliche Funktionen aufgezeigt werden, die ein Linux on z im Gegensatz zu einem Linux on x86 bietet.

Der Hauptteil fuehrt zuerst in Linux unter x86 und Linux on z ein und geht dann ueber zu den Unterschieden zwischen den beiden.
Der Schlussteil fasst in einem Fazit die oben genannten Fragen zusammen.
